\textbf{Forces:}

\begin{itemize} \itemsep2pt
\item Narator
\item Bass flute
\item Harp
\item Percussion I \& II (identical setup for each):
    \begin{itemize}
    \item triangle, slate, glockenspiel
    \item planks of purpleheart (3 sizes)
    \item brake drum, bass drum (large mallet, sponge, superball)
    \item tam-tam (large)
    \end{itemize}
\item Viola
\item Cello I \& II
\item Contrabass I \& II
\end{itemize}

\textbf{Accidentals.} Accidentals govern only one note. This is true even for
successive noteheads at the same staff position. The sequence of G$\sharp$4
followed by G4 (without accidental) is to be understood as G$\sharp$4 followed
by G$\natural$4.

\textbf{Appoggiatura figures.} Play runs of small-note appoggiaturas as fast as
possible starting directly on the beat and then landing on the full-size note
shown below.

\textbf{Bass flute.} The bass flute sounds an octave lower than written. Play
passages marked ``covered'' (or ``cov.'') by covering the opening of the flute
with the lips; such passages sound an octave plus a  minor seventh lower than
written. The two multiphonics (``L.5'' and ``L.42'') are numbers 5 and 42 in
Carin Levine's book \textit{Die Spieltechnik der Fl\"{o}te}, volume II. Trills
without secondary pitches are color trills. Transitions between tone (T) and
air (A) are shown with arrows.

\textbf{Harp.} Play passages marked ``whisk'' by running the fingernail (or a
coin or plastic guitar pic) laterally up one of the strings of the harp to
create a whisking sound. The sound is usually paired with a scraping sound
played by the percussionists on slate; seek to blend well with the sound of the
percussionists' stones. Rehearsal mark J features an extended passage for the
harpist to bow the F$\sharp$3 and G$\flat$ strings with a pair of cello (or
violin bows). Treat the passage as a type of very slow-moving cadenza in
beating tones and color changes in acknowledgement of \'{E}liane Radigue's work
on the technique.

\textbf{Strings.} No scordatura. Passages marked with a damp symbol should be
played with the left hand damping the string at the position indicated: lightly
lay three fingers on the string to produce a beautiful grey sound with
perceptible (but muted) sense of pitch. The instruction \textbf{XFB}
(``extremely fast bow'') should be played as an extremely fast, extremly light,
extremely irregular type of tremolo flautando: use very generous amounts of bow
(to create extremely fast bow strokes) and change the bow frequently in a
constantly irregular rhythm. The aural result of the technique is a
``fluorescent'' type of flautando that brings out the upper partials of the
string's sound. Triple-staccati indicate rimbalzandi: aim for three bounces of
the bow per note head. Play passages notated on a 1-line staff directly on the
wood of the bridge: white noise results with almost no sense of pitch.
Transitions between ponticello (P), ordinario (O) and tasto (T) string contact
points are shown with arrows; P1, P2, P3, P4 indicate string contact points
progressively closer to the bridge (and brighter and more acidic in timbre).
Important: passages marked ``bisb.'' or ``quasi bisb.'' in the strings are to
be played in imitation of harp bisbigliandi: cycle through the pitches
indicated as quickly as possible; note that these passages are an effect of the
left hand (and \textbf{not} a right-hand technique). White diamond noteheads
indicate natural harmonics in the usual way; black diamond noteheads
indicate half harmonic pressure.
