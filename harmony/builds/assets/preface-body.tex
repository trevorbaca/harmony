\textbf{Forces:}

\begin{itemize} \itemsep2pt
\item Narator
\item Bass flute
\item Percussion I \& II (identical setup for each):
    \begin{itemize}
    \item triangle, slate, glockenspiel
    \item planks of purpleheart (3 planks each, all relatively close in pitch)
    \item brake drum, bass drum (large mallet, sponge, superball)
    \item tam-tam (large)
    \end{itemize}
\item Harp
\item Viola
\item Cello I \& II
\item Contrabass I \& II
\end{itemize}

\textbf{Accidentals.} Accidentals govern only one note. This is true even for
successive noteheads at the same staff position. The sequence of G$\sharp$4
followed by G4 (without accidental) is to be understood as G$\sharp$4 followed
by G$\natural$4.

\textbf{Appoggiaturas.} Play runs of small-note appoggiaturas as fast as
possible starting directly on the beat; land immediately on the full-size note
shown below and sustain to the end of the duration indicated.

\textbf{Flat glissandi.} Flat glissandi are sometimes used as a typographical
variant of ties.

\textbf{Metric modulations.} One hundred three of the metric modulations in the
music are indicated with spanners (marked ``MM''). These are included as a
signal to the conductor that a given part may be used aurally to check a
modulation.

\textbf{Bass flute.} The bass flute sounds an octave lower than written. Play
passages marked ``covered'' (or ``cov.'') by covering the opening of the flute
with the lips; such passages sound an octave plus a  minor seventh lower than
written. The two multiphonics (``L.5'' and ``L.42'') are bass flute
multiphonics 5 and 42 in Carin Levine's book \textit{Die Spieltechnik der
Fl\"{o}te}, volume II. Trills without secondary pitches are color trills.
Transitions between tone (T) and air (A) are shown with arrows.

\textbf{Percussion.} Pieces of slate may be found at a hardware or flooring
store; select pieces that are both about a foot square; select pieces with
slightly different surface irregularities (and resulting timbre); the two piece
should sound slightly detuned from one another. The score gives two different
ways of playing the slate. Scrape the slate in a semicircular motion that
traverses the width of the slate in the duration indicated: ``scrape'' written
above a quarter note takes twice as long to travel the same distance as
``scrape'' written above an eighth note (and sounds correspondingly faster).
Brush the slate with a toothbrush or other stiff-bristled brush.
\textbf{Purpleheart.} Planks of purpleheart may be found at hardware and
flooring stores. Each percussion part requires three pieces of purpleheart
(high, middle, low) corresponding to the three-line staff in the score. The
three high, middle, low pitches should all be relatively close to each other
(within about a major third). Additionally, \textit{the two 3-piece sets of
purpleheart should be mircotonally detuned from each another.} Thus the two
``high'' planks must almost (but not quite) match each other in pitch; likewise
the two ``middle'' planks must almost (but not quite) match each other, and the
two ``low'' planks must almost (but not quite) match each other. The goal is a
six-note swarm of pitches that loose some of their distinctiveness when sounded
together. \textbf{Tam-tam.} The two tam-tams should match in pitch. Play with
an attackless roll that privileges the fundamental and suppresses the upper
partials. Move the place of attack slowly from the rim to within a few inches
of the center (and back) \textit{ad lib} throughout the piece, even though
these transitions are not yet shown in the score. \textbf{Brake drum.} Play
passages marked ``brake drum (papertowel)'' by drawing a dry papertowel in a
continuous course over the rough metallic surface of the drum; the resulting
sound is a strikingly disembodied white noise.

\textbf{Harp.} Play passages marked ``whisk'' by running the fingernail (or a
coin or plastic guitar pic) laterally up one of the harp's strings to create a
whisking sound. The sound is usually paired with the percussionists' scraped
pieces of slate. \textbf{Bowing the harp.} Rehearsal marks J and BB feature
passages for the harpist to bow the instrument with a pair of cello (or violin)
bows. Use a single bow (RH) where only a single pitch is notated; use a pair of
bows (RH and LH together) where two pitches are notated. Bow the first string
at a string contact point that brings out the seventh partial; bow the second
string with changes in speed that effect the beating patterns given in the
score (``8 pul. / beat'' meaning 8 pulses per slow quarter-note beat, for
example, accomplished as slight differences in speed at which each string is
bowed). Both such passages in the piece should be beautiful; treat the music as
slow-moving color cadenzas in acknowledgement of \'{E}liane Radigue's work on
the technique.

\textbf{Strings.} No scordatura. The viola and cellos sound as written. The
contrabass sounds an octave lower than written in the bass clef; \textit{the
contrabass sounds as written in the trble clef}. \textbf{LH damping.} Passages
marked with a damp symbol should be played with the left hand damping the
string at the position indicated: lightly lay three fingers on the string to
produce a beautiful grey sound with perceptible (but muted) sense of pitch.
\textbf{XFB.} Passages marked ``XFB'' (``extremely fast bow'') should be played
with a fast, extremly light, desynchronized type of tremolo flautando: use
generous amounts of bow and change the bow irregularly (while noting that the
technique is decidedly less hectic than it might first appear because the bow
only skims the surface of the string throughout: do not play ``into'' the
string at all). Most XFB passages seem to be helped by playing somewhat tasto
on the string. The aural result of the technique is a ``fluorescent'' type of
flautando that brings out the middle partials of the string's sound.
\textbf{Rimbalzandi.} Triple-staccati indicate rimbalzandi: aim for three
bounces of the bow per note head. \textbf{Playing directly on the wood of the
bridge.} Play passages notated on the 1-line staff directly on the wood of the
bridge: white noise results with almost no sense of pitch. \textbf{String
contact point (SCP) transitions.} Transitions between ponticello (P), ordinario
(O) and tasto (T) string contact points are shown with arrows; P1, P2, P3, P4
indicate string contact points progressively closer to the bridge (and brighter
and more acidic in timbre); T1, T2, T3, T4 indicate string contact points
progressively closer to the nut (and mellower and smoother in timbre).
\textbf{Quasi bisbigliandi.} Passages marked ``quasi bisb.'' in the strings are
to be played in imitation of harp bisbigliandi: cycle through the pitches
indicated as quickly as possible; note that these passages are an effect of the
left hand (and \textbf{not} a type of right-hand tremolo, even though the
technique is marked with three hash signs). \textbf{Harmonics and
half-harmonics.} White diamond noteheads indicate natural harmonics in the
usual way; black diamond noteheads indicate half harmonic pressure.
\textbf{Half col legno tratto.} Play passages marked ``1/2 clt'' with the bow
rotated to allow both hair and wood to travel across the string. The goal is to
introduce a healthy amount of whisking into the sound, especially when combined
with full up-bow and full down-bow strokes. \textbf{Full-bow strokes.} Up-bow
and down-bow symbols equipped with dangling tails indicate complete bow strokes
in the direction given. The symbols provide for very fast movements of the bow,
usually played half col legno tratto.
